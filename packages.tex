% zum Einblenden von Grafiken
\usepackage{graphicx}
% ändert die automatisch generierten wörter (inhaltsverzeichnis etc.) zu deutsch. 
% default ist englisch, also einfach das includepackage löschen.
\usepackage[english, ngerman]{babel}
%für anführungszeichen im text
\usepackage[autostyle, german=guillemets, german=quotes]{csquotes}
% um Grafiken gezielt anzuzeigen mit /begin{figure}[H] /end{figure}
\usepackage{float}
%Quelltexthighlighting, integration dessen nicht als rastergrafik, sondern skalierbar
\usepackage{minted}
% für hintergrundbilder
\usepackage{eso-pic}
% um hintergrund transparent zu machen
\usepackage{transparent}
%für verschiedene hintergrundbilder bei gerade/ungeraden seitenzahlen
\usepackage{ifthen}
%für tabellen 
\usepackage{tabularx}

