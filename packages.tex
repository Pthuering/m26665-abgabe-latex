% zum Einblenden von Grafiken
\usepackage{graphicx}
% ändert die automatisch generierten wörter (inhaltsverzeichnis etc.) zu deutsch. 
% default ist englisch, also einfach das includepackage löschen.
\usepackage[english, ngerman]{babel}
% für anführungszeichen im text
\usepackage[autostyle, german=guillemets, german=quotes]{csquotes}
% um Grafiken gezielt anzuzeigen mit /begin{figure}[H] /end{figure}
\usepackage{float}
%Quelltexthighlighting, integration dessen nicht als rastergrafik, sondern skalierbar
\usepackage{minted}
% für hintergrundbilder
\usepackage{eso-pic}
% um hintergrund transparent zu machen
\usepackage{transparent}
% für verschiedene hintergrundbilder bei gerade/ungeraden seitenzahlen
\usepackage{ifthen}
% für tabellen 
\usepackage{tabularx}
% literaturverzeichnisse verwenden
\usepackage[style=authortitle,natbib,backend=biber]{biblatex}
% formatierung der footnote citation nach singers gemüt
\makeatletter%
\long\def\@makefntext#1{%
    \parindent 1em\noindent \hb@xt@ 1.8em{\hss \hbox{{\normalfont \@thefnmark}}. }#1}%
\makeatother
% link einfärben und zur referenz springen
\usepackage{hyperref}
\hypersetup{
    colorlinks = true,
    linkcolor=black,
    citecolor=black,
    urlcolor=blue,
    pdfborder= 0 0 0 % entfernt rahmen, falls noch vorhanden
}
% referenzieren eines kapitels mit namen
\usepackage{nameref}