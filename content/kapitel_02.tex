\section{Kapitel 2} \label{kapitel_02}
you already know. wow. 
\begin{listing}[h]
    %caption wird benötigt, um das lisiting in das verzeichnis aufzunehmen, same mit abbildungen
    \caption[Quelltext]{beschreibe den Quelltext hier.}
% linenos -> numeriert zeilen, frame=lines -> grenzt code mit linien ab, fontsize -> schriftgröße, 
% firstnumber -> zeile, in der snippet anfängt, muss al erster parameter stehen
% unter den mathescape eingaben dann die gewählte sprache in {}
% verfügbare sprachen via pygmentize dokumentation 
\begin{minted}
    [
        firstnumber=5,
        frame=lines,
        framesep=2mm,
        baselinestretch=1.2,
        linenos
        ]
{java} 

package com.tutego.insel.ds.observer;

import java.util.*;

public class Party
{
  public static void main( String[] args )
  {
    Observer achim    = new JokeListener( "Achim" );
    Observer michael  = new JokeListener( "Michael" );
    JokeTeller chris  = new JokeTeller();

    chris.addObserver( achim );

    chris.tellJoke();
    chris.tellJoke();

    chris.addObserver( michael );

    chris.tellJoke();

    chris.deleteObserver( achim );

    chris.tellJoke();
  }
}

\end{minted}
    \label{lst:mein_Quelltext}
\end{listing}

Im Quelltext \ref{lst:mein_Quelltext} sieht man einen Ausschnitt eines Java Quelltextes.