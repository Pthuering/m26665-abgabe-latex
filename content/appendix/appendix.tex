\appendix 

\section*{Anhang}   \label{chap:appendix}
\addcontentsline{toc}{section}{Anhang}


% um numerierung im anhang trotzdem zu erhalten
\renewcommand{\thesubsection}{\Alph{subsection}}
% notation von abbildungen wird geänder: 1 > A.1 
\renewcommand{\thefigure}{\thesubsection.\arabic{figure}}
\renewcommand{\theHfigure}{\thesubsection.\arabic{figure}}%<---!!!!--- Erklärung: https://tex.stackexchange.com/questions/442514/ref-jumps-not-to-the-right-figure-in-appendix
% an dieser stelle würde ich gern die notation von listings im anhang anpassen, bisher aber leider erfolglos

% berichtigen des counters für abbildungen
\setcounter{figure}{0}

\subsection{Nachbereitung}

Als allererstes, hier noch die letzte Referenz zum folgenden Listing im Anhang: \ref{lst:helloworld}  
Die hier benutzte Arbeit entstand im Rahmen des Moduls \textit{Wissenschaftliches Arbeiten} aus dem 
4. Semester. Da keine Bilder oder Grafiken vorgesehen waren, habe ich ein paar eher weniger relevante 
eingefügt. Trotzdem hoffe ich, dass der Leseflow der Arbeit einigermaßen angenehm ist. Alle für diese 
Abgabe relevanten Inhalte sind gleichmäßig über die Arbeit verteilt und können über die Seite 'Hinweise' 
angewählt werden. Überflüssige stellen der Arbeit wurden gelöscht, um den Inhalt bündig beisammen zu halten 
und nicht zu dünn zu verstreuen. Deshalb könnte der Text bei genauerem lesen allerdings auch teilweise 
nur wenig Sinn ergeben. 

    % großes H als kommando, damit content genau dort angezeight wird
    \begin{listing}[H]
        %caption wird benötigt, um das lisiting in das verzeichnis aufzunehmen, same mit abbildungen
        \caption[Java Hello World Beispiel]{Every java dev says 'Hello world!', but noone ever says 'How are you doing, world?'...}
    % linenos -> numeriert zeilen, frame=lines -> grenzt code mit linien ab, fontsize -> schriftgröße, 
    % firstnumber -> zeile, in der snippet anfängt, muss al erster parameter stehen
    % unter den mathescape eingaben dann die gewählte sprache in {}
    % verfügbare sprachen via pygmentize dokumentation 
    \begin{minted}
        [
            firstnumber=5,
            frame=lines,
            framesep=2mm,
            baselinestretch=1.2,
            linenos
            ]
    {java} 
    
        /*
        Java Hello World example.
        */
            
        public class HelloWorldExample{
            
          public static void main(String args[]){
            
            /*
            Use System.out.println() to print on console.
            */
            System.out.println("Hello World !");
            
          }
            
        }
            
        /*
        OUTPUT of the above given Java Hello World Example would be :
        Hello World !
        */
    
    \end{minted}
        \label{lst:helloworld}
    \end{listing}


\setcounter{figure}{0}


Hier noch die letzte Anforderung an das Dokument in Form eines Bildes von unserem Hochschullogo im Anhang.

Lorem ipsum dolor sit amet, consetetur sadipscing elitr, sed diam nonumy eirmod tempor invidunt ut 
labore et dolore magna aliquyam erat, sed diam voluptua. At vero eos et accusam et justo duo dolores et 
ea rebum. Stet clita kasd gubergren, no sea takimata sanctus est Lorem ipsum dolor sit amet. Lorem ipsum 
dolor sit amet, consetetur sadipscing elitr, sed diam nonumy eirmod tempor invidunt ut labore et dolore 
magna aliquyam erat, sed diam voluptua. At vero eos et accusam et justo duo dolores et ea rebum. Stet 
clita kasd gubergren, no sea takimata sanctus est Lorem ipsum dolor sit amet. Lorem ipsum dolor sit
amet, consetetur sadipscing elitr, sed diam nonumy eirmod tempor invidunt ut labore et dolore magna
aliquyam erat, sed diam voluptua. At vero eos et accusam et justo duo dolores et ea rebum. Stet 
clita kasd gubergren, no sea takimata sanctus est Lorem ipsum dolor sit amet.   

 
    \begin{figure}[H]
        \centering
        \includegraphics[width=0.9\textwidth]{graphics/HSHARZ.png}
        \caption[Hochschullogo]{Das Logo unserer Hochschule}
        \label{fig:appendix_logo}
    \end{figure}


    % links aus text und aus den verzeichnissen funktionieren nun normal, allerdings nur bei nutzung von \ref
    % \nameref führt immer noch zum falschen ergebnis