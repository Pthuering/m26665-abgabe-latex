\section{Was ist ein Problem?}

Diese Frage selbst stellt schon ein Problem dar. Konkreter sollten wir uns also Fragen: 
“Wie sieht ein Problem aus, welches ich mit der Ingenieurmethodik lösen kann?” 

Einfach ausgedrückt: Ein System mit ungewissen Attributen befindet sich in einem Zustand A und 
soll mit gegeben Ressourcen in einen besseren Zustand B versetzt werden. 

Um diese recht einfache Definition im Kern zu verstehen müssen jedoch einige Elemente in 
einen dringend notwendigen Kontext gestellt werden. So müssen wir, um eine Problematik mit 
der Ingenieurmethodik auf zu arbeiten folgende Begriffe definieren: Veränderung, Ressourcen, 
am Besten \& Ungewiss.

    \subsection{Veränderung}

    Veränderung ist der Übergang von einer Situation A in eine Situation B wobei sich Situation A von B 
    unterscheidet. In unserem Falle sind Situation A und B allerdings nicht fest definiert sondern müssen 
    erst erkannt werden. Herauszufinden wie Situation B aussieht und wie der Übergang herbei zu führen ist
    ist Aufgabe des Ingenieurs. Hierzu muss er allerdings Situation A möglichst genau definieren, eine
    Aufgabe die ihm selten vom Auftraggeber abgenommen wird, auch wenn er dies eigentlich tun sollte.
    So muss der Ingenieur erstmal das genaue Problem feststellen um dann für eben dieses Problem nicht 
    die Lösung sondern eine Lösung zu finden. Eben diese Lösung sollte dann die passendste aller
    möglichen Lösungen sein.

    \subsubsection{Ressourcen}

    Viele unterschiedliche Probleme benötigen viele unterschiedliche Ressourcen. Wie diese Ressourcen 
    aussehen ist daher immer vom Problem abhängig. Sie können materialistischer Natur sein, so gestaltet 
    sich der Bau einer Brücke als ungemein schwieriger wenn kein Zugriff auf Stahl, Beton und Werkzeuge 
    besteht. Aber auch Geld und Zeit sind hier als Ressource zu betrachten. Mit einer zu kurzen Frist 
    oder ohne monetäre Mittel ist der Bau einer Brücke undenkbar. Zu guter Letzt ist auch das Personal 
    und dessen Wünsche als Ressource die gemanaged werden muss an zu sehen.

    Eine Faustregel um herauszufinden was als Ressource betrachtet werden muss ist wie folgt: Wenn 
    aus zwei Ausgangslagen, die, bis auf einen Faktor, gleich sind zwei unterschiedliche Ergebnisse, 
    eines von beiden dem anderen zu bevorzugen, hervorgehen ist dieser eine Faktor als Ressource 
    anzusehen. Eine Arbeitsgruppe mit 10 Minuten Zeit findet wahrscheinlich eine weniger durchdachte 
    Lösung als eine Arbeitsgruppe mit einer zwei Wochen Frist. Jedoch können damit auch selten bedachte
    Aspekte einfließen. Ein Team aus erfahren Ingenieuren bringt höchstwahrscheinlich ein anderes 
    Ergebnis zustande als ein paar Neulinge auf dem entsprechenden Gebiet.

    Der letzte wichtiger Aspekt der Ressourcen ist die Austauschbarkeit untereinander. So kann mit Geld 
    Personal und Material gekauft werden, aus Zeit kann Wissen in Form von zusätzlicher Recherche geschöpft
    werden und ein kleiner Zeitverlust in Form einer Kaffeepause kann zu einem stark erhöhten 
    Enthusiasmus im Team führen. Die Aufgabe des Ingenieurs ist es all diese Ressourcen und deren 
    Möglichkeiten im Überblick zu behalten und sie bestmöglich in die Problemlösung mit zu integrieren.

    \subsubsection{Am Besten}

    Der Begriff “Gut” beziehungsweise “am Besten” ist im Falle des Ingenieurs nicht die ideale 
    Lösung des Problemes in einem ansonsten leeren Raum sondern ein möglichst optimaler Kompromiss 
    aus allen Faktoren und Kriterien. Da eben diese Kriterien von Person zu Person zumeist 
    unterschiedlich stark gewichtet werden ist es die Aufgabe des Ingenieurs die allgemeine 
    Gewichtung eben dieser Kriterien durch die Gesellschaft zu ermitteln. Da auch dies ein subjektiver 
    Prozess ist (bei dieser Evaluation  spielen die persönlichen Kriterien des Ingenieurs natürlich 
    auch eine Rolle) kann es keine ideale Lösung geben. Die “beste” Lösung die im Laufe dieses 
    Prozesses zu finden ist ist die, die möglichst viele Kriterien einfließen lässt und abhängig 
    der Gewichtung möglichst viele Teilnehmer zufriedenstellt.

    \subsubsection{Ungewissheit}

    Ein großer Teil, nein, gar der Kern eines jeden Ingenieurs-Projektes ist es mit unvollständigen 
    Informationen auf ein schwammig definiertes Ziel mit unbekannt vielen Faktoren und Möglichkeiten 
    zu zu arbeiten. Wäre dies anders, also die Informationen vollständig, die Aufgabe klar definiert 
    und alle Faktoren bekannt wäre das Ganze nicht mehr als eine wissenschaftliche Formel zu lösen. 
    Alle Variablen werden Eingesetzt und am Ende steht das Ergebnis. Sollte sich im Laufe dessen 
    rausstellen dass sich Fehler eingeschlichen haben kann man diese korrigieren und die Variablen 
    nochmals einsetzen. Mögen sich eben diese Variablen auch ändern, die Formel und Prozedur bleiben 
    gleich. Eben diese Sicherheit ist bei einem Ingenieur-Problem jedoch nicht gegeben, da bis zum 
    Schluss kein vollständiges Bild der Ausgangs- und Endlage bekannt ist.
    Die Aufgabe des Ingenieurs ist es also diese Ungewissheit der Situation zu minimieren, in dem er 
    möglichst viele Faktoren und Kriterien in seine Überlegungen mit einfließen lässt und so einen 
    optimalen Kompromiss findet.

    \subsection{Der Kern des Ingenieursproblems}

    Zusammengefasst sieht jedes Problem der Ingenieurmethodik wie folgt aus: Ein System mit 
    ungewissen Attributen befindet sich in einem Zustand A und soll mit gegebenen Ressourcen in 
    einen besseren Zustand B versetzt werden.


