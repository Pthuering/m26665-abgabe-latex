\section{Fazit}

Betrachten wir nun nochmals den Prozess in seiner Gesamtheit. Es gibt ein Problem in der Welt und 
dieses wollen wir lösen. Mit dem Wissen niemals alles wissen zu können fangen wir also an 
Informationen zu sammeln um uns ein zumindest teilweise vervollständigtes Bild der Problematik zu 
machen. Wir betrachten das Problem von allen Seiten und machen aus, welche Kriterien dazu beitragen 
würden, es zu lösen. Wir legen uns ein Ziel fest, stellen Hypothesen auf wie wir dieses erreichen 
und führen dann Tests durch um zu sehen ob eben diese Hypothesen stimmig sind. Kommen wir so auf 
ein Ergebnis prüfen wir nochmals ob dieses auch wirklich eine Lösung zu dem Problem darstellt und 
holen uns dann die Meinung Anderer ein um nochmals sicher zu gehen. 

Nüchtern betrachtet ist dies keine neue, revolutionäre Herangehensweise. Ganz im Gegenteil, 
diese Art der Problemlösung wird schon von Kindern intuitiv verwendet. Jedoch ist diesen 
Kindern, und anbei auch vielen Erwachsenen, nicht bewusst wie viele Prozesse und logische 
Schlüsse hinter dieser Methodik stehen. Diese Intuition ist es auch, die dafür sorgt, dass diese 
Prozeduren bis heute in diesem Maße befolgt werden. Gerechtfertigt wird die Methodik des Ingenieurs 
durch Jahrhunderte an Erfahrung und einem sich selbst kontrollierenden System, welches flexibel 
genug ist sich selbst zu korrigieren. Solange man der Lehre der Heuristik, also der Kunst mit 
unvollständigen Informationen und wenig Zeit dennoch zu wahrscheinlichen Aussagen oder praktikablen 
Lösungen zu kommen, folgt ist die hier beschriebene Methodik des Ingenieurs eine valide Handlungsform 
und für den Fall, dass man eben diese Heuristik anzweifelt ist der Erfolg der Ingenieurmethodik 
doch nicht von der Hand zu weisen.