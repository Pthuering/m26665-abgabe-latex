\section{Hypothese}

(kursiv)
Ziel der Hypothese ist es, eine Lösung für die Projektaufgabe zu finden. Darin beinhaltet 
sind außerdem Subziele, Hypothesen, Faktoren und Performance Kriterien, um die Richtigkeit der 
Lösung zu testen.

Alleinstehend ist die Hypothese nicht mehr als ein spekulativer Vorschlag, der für wissenschaftliche 
Vorgehen nicht ausreichend ist. Erst mit Hilfe der folgenden Schritte der wissenschaftlichen Methode, 
Synthese und Validierung, kann die Hypothese legitimiert werden.

    \subsection{Lösung spezifizieren}

    Die Lösung besteht aus einem Mechanismus und einer Prozedur. Es kann sein, dass der Mechanismus
    bereits besteht. In diesem Fall muss er nur noch gekauft werden. Soll zum Beispiel die Länge 
    eines Tisches gemessen werden, wäre ein Maßband der optimale Mechanismus. Prozedur hierbei 
    wäre dann der Messvorgang. Wird eines von beiden neu entwickelt, müssen alle dafür nötigen 
    Schritte in der Spezifizierung dokumentiert werden, inklusive benötigter Skizzen, Schemata, 
    Rezepten, Pseudocode, Algorithmen, etc. Für die bereits bestehenden Komponenten ist eine kurze
    Beschreibung in Begleitung von Spezifikationen des Herstellers ausreichend. 
    Um diesen Schritt abzusichern, können auch mehrere Lösungen entworfen und angewendet werden, 
    solange diese im Rahmen des Budgets liegen.

    \subsection{Ziele setzen}

    Jede Aufgabe hat mindestens ein Ziel, um die Reaktion der Task Unit bei Anwendung der Lösung zu 
    beurteilen. Ähnlich des Aufgabenbaumes ist es sinnvoll, auch einen Zielbaum zu erstellen, um 
    das übergeordnete Ziel in kleine Ziele aufzuteilen. Zu jedem Ziel sollte jeweils nur eine 
    korrespondierende Aufgabe bzw. ein Experiment zugeordnet werden.
    Zu jedem Ziel sollte auch gleichzeitig eine zugehörige Hypothese aufgestellt werden. Diese 
    drückt das erwartete Resultat der Zusammenfassung des Ziels aus. Hypothesen sollten, wenn 
    möglich, neutral formuliert werden, um Voreingenommenheit gegenüber dem Ergebnis zu vermeiden. 
    Völlige Neutralität in der Hypothese muss allerdings nicht erzwungen werden, da diese zu einem 
    Späteren Zeitpunkt, erneut durch eine Nullhypothese überprüft wird. 
    Eine kleine Zahl von Experimenten kann manchmal schon ausreichen, um eine Vielzahl von Zielen 
    zu erreichen. Im Regelfall reicht es allerdings auch aus, wenn ein Experiment zur Erfüllung 
    eines Ziels führt. Extreme Versuche zur Optimierung  sollten mit Bedacht unternommen werden.

    \subsection{Faktoren Definieren}

    Der erste Schritt beim Definieren der Faktoren sollte sein, eine vollständige Liste aller 
    Parameter und Bedingungen für das Projekt anzufertigen. Diese sollte am ersten Tag des Projekts 
    angefangen und im Laufe dessen weiterhin fortgeführt werden. Diese Liste kann mit einer Handvoll 
    Faktoren anfangen und bis hin zu mehreren tausend Einträgen wachsen. 
    Bevor fortgefahren werden kann, sollte sich jeder Mitarbeiter ein eingehendes Verständnis dieser 
    Liste aneignen. 
    Die Zuweisung von Werten für die Parameter hängt oft von der eigenen Vernunft ab. Zum Beispiel 
    sollte ein Experiment, welches die Performance der Teilnehmer bei einer bestimmten Aufgabe misst, 
    bei ähnlichen, bestenfalls auch gleichen, Lichtverhältnissen stattfinden, wie sie im normalen 
    Umfeld der Kohorte vorherrschen. Für schwerer festzulegende Faktoren kann ein sogenannter 
    Operating-Point Pilot durchgeführt werden. Die Teilnehmer des OP Pilots sind allerdings nicht 
    mehr zur Teilnahme am eigentlichen Experiment berechtigt, weshalb man darauf achten sollte, 
    keinen zu großen Teil der von Beginn an schon kleinen Kohorte für vorbereitende Experimente zu nutzen.

    \subsection{Performance Metriken aufstellen}

    Performance Metriken sind ein Postulat, dass die Ergebnisse der Aufgabe in Leistungsmaßstäbe 
    umwandelt, damit daraus Schlussfolgerungen über das Aufgabenziel gezogen werden können. Eine 
    Faustregel ist, ein möglichst sinnvolles Medium zur Darstellung der aus den Experimenten gewonnenen 
    Daten zu wählen. Gute Performance Metriken sind unter anderem Histogramme oder Pareto Diagramme. 
    Die Auswahl der richtigen Performance Metrik kann den Unterschied zwischen einer gescheiterten 
    und einer erfüllten Aufgabe ausmachen. Das Ändern oder Austauschen von Performance Metriken 
    sollte jedoch nie ohne guten Grund und auf gar keinen Fall auf Kosten von Objektivität 
    passieren. Repräsentiert eine Performance Metrik die gesammelten Daten zum Beispiel kompakter, 
    aber trotzdem detaillierter als eine andere, wäre ein wechsel durchaus angemessen. Jede 
    Aufgabe muss mindestens eine Sub Aufgabe haben, deren Ziele messen, wann und wie gut die 
    Primäre Aufgabe erreicht wurde.


