\section{Was ist ein Problem?} \label{chap:1}

% Paragraph und Subparagraph sind hier wie 
Diese Frage selbst stellt schon ein Problem dar. Konkreter sollten wir uns also Fragen: 
“Wie sieht ein Problem aus, welches ich mit der Ingenieurmethodik lösen kann?” 
\paragraph{Einfach ausgedrückt:} 

Ein System mit ungewissen Attributen befindet sich in einem Zustand A und 
soll mit gegeben Ressourcen in einen besseren Zustand B versetzt werden. 
\subparagraph{Um diese recht einfache Definition} 

im Kern zu verstehen müssen jedoch einige Elemente in 
einen dringend notwendigen Kontext gestellt werden. So müssen wir, um eine Problematik mit 
der Ingenieurmethodik auf zu arbeiten folgende Begriffe definieren: Veränderung, Ressourcen, 
am Besten \& Ungewiss.

    \subsection{Veränderung}

    Veränderung ist der Übergang von einer Situation A in eine Situation B wobei sich Situation A von B 
    unterscheidet. In unserem Falle sind Situation A und B allerdings nicht fest definiert sondern müssen 
    erst erkannt werden. Herauszufinden wie Situation B aussieht und wie der Übergang herbei zu führen ist
    ist Aufgabe des Ingenieurs. Hierzu muss er allerdings Situation A möglichst genau definieren, eine
    Aufgabe die ihm selten vom Auftraggeber abgenommen wird, auch wenn er dies eigentlich tun sollte.
    So muss der Ingenieur erstmal das genaue Problem feststellen um dann für eben dieses Problem nicht 
    die Lösung sondern eine Lösung zu finden. Eben diese Lösung sollte dann die passendste aller
    möglichen Lösungen sein.


    \subsubsection{Am Besten}

    %hier handelt es sich um ein indirektes Zitat, weshalb in der Fußnote ein 'Vgl.' nicht fehlen darf
    Der Begriff “am Besten” ist im Falle des Ingenieurs nicht die ideale 
    Lösung des Problemes in einem ansonsten leeren Raum sondern ein möglichst optimaler Kompromiss 
    aus allen Faktoren und Kriterien. Da eben diese Kriterien von Person zu Person zumeist 
    unterschiedlich stark gewichtet werden ist es die Aufgabe des Ingenieurs die allgemeine 
    Gewichtung eben dieser Kriterien durch die Gesellschaft zu ermitteln. Da auch dies ein subjektiver 
    Prozess ist (bei dieser Evaluation  spielen die persönlichen Kriterien des Ingenieurs natürlich 
    auch eine Rolle) kann es keine ideale Lösung geben. Die “beste” Lösung die im Laufe dieses 
    Prozesses zu finden ist ist die, die möglichst viele Kriterien einfließen lässt und abhängig 
    der Gewichtung möglichst viele Teilnehmer zufriedenstellt. \footnote{Vgl. \cite[S. 35]{bock2001}}
       

    \subsubsection{Ungewissheit}

    %dies ist ein direktes zitat, weshalb das Vgl. in der Fußnote wegfällt
    \begin{quote}
        Ein großer Teil, nein, gar der Kern eines jeden Ingenieurs-Projektes ist es mit unvollständigen 
    Informationen auf ein schwammig definiertes Ziel mit unbekannt vielen Faktoren und Möglichkeiten 
    zu zu arbeiten. Wäre dies anders, also die Informationen vollständig, die Aufgabe klar definiert 
    und alle Faktoren bekannt wäre das Ganze nicht mehr als eine wissenschaftliche Formel zu lösen. 
    Alle Variablen werden Eingesetzt und am Ende steht das Ergebnis. Sollte sich im Laufe dessen 
    rausstellen dass sich Fehler eingeschlichen haben kann man diese korrigieren und die Variablen 
    nochmals einsetzen. Mögen sich eben diese Variablen auch ändern, die Formel und Prozedur bleiben 
    gleich. Eben diese Sicherheit ist bei einem Ingenieur-Problem jedoch nicht gegeben, da bis zum 
    Schluss kein vollständiges Bild der Ausgangs- und Endlage bekannt ist.
    Die Aufgabe des Ingenieurs ist es also diese Ungewissheit der Situation zu minimieren, in dem er 
    möglichst viele Faktoren und Kriterien in seine Überlegungen mit einfließen lässt und so einen 
    optimalen Kompromiss findet. \footnote{ \cite[S. 112]{koen2013}}
    \end{quote}
     


    \subsection{Der Kern des Ingenieursproblems}

    % hier wird von einer Webseite zitiert, deren Link im Literaturverzeichnis abgerufen werden kann
    Zusammengefasst sieht jedes Problem der Ingenieurmethodik wie folgt aus: Ein System mit 
    ungewissen Attributen befindet sich in einem Zustand A und soll mit gegebenen Ressourcen in 
    einen besseren Zustand B versetzt werden. \footnote{ \cite{method}}


