\section{Synthese}

Das Ergebnis der Synthesephase sollte die Anwendung der designten Lösung sein, um die gesetzten Ziele 
zu erreichen und die die Aufgabe begleitende Hypothese zu validieren.

Auch dieser Prozess kann in 4 kleinere Funktionen unterteilt werden aber anders als in der 
Analysephase, muss die Reihenfolge aller Schritte genau eingehalten werden. In dieser Phase werden 
alle vorher aufgestellten Thesen, Faktoren, Bedingungen, etc. mit Hilfe von Experimenten in die 
Praxis umgesetzt, um erste konkrete Ergebnisse zu erzielen.


    \subsection{Lösungsansätze implementieren}

    Es gibt zwei Arten, auf die eine Lösung implementiert werden kann, entwickeln und kaufen. 
    Ein Beispiel für eine selbst entwickelte Lösung kann im Listing \ref{lst:loesimpl} betrachtet werden.
    Sollte es den Anforderungen entsprechen, ist es fast immer effektiver die Lösung zu erwerben, 
    als sie selbst zu entwickeln. Dies trifft auch zu, sollte die erworbene Komponente nur zu einem 
    Teil zur Lösung beitragen. 
    Besteht ein Teil der Lösung aus einer Datenbank, sollte der Aufwand betrieben werden, um die Daten 
    vollständig, roh und unbearbeitet zu erhalten. Dies trifft vor allem zu, wenn die Daten von einer 
    anderen Forschungsgruppe oder Firma kommen. Häufig treten im Zusammenhang dessen auch politische 
    oder Eigentums Probleme auf. Von Drittparteien erhaltene Daten sollten generell immer mit einer 
    gewissen Skepsis behandelt werden.


    \begin{listing}[H]
        %caption wird benötigt, um das lisiting in das verzeichnis aufzunehmen, same mit abbildungen
        \caption[Loesungsimplementation Codebeispiel Java]{Beispiel einer Loesungsimplementation nach den festgelegten Vorschriften}
    % linenos -> numeriert zeilen, frame=lines -> grenzt code mit linien ab, fontsize -> schriftgröße, 
    % firstnumber -> zeile, in der snippet anfängt, muss al erster parameter stehen
    % unter den mathescape eingaben dann die gewählte sprache in {}
    % verfügbare sprachen via pygmentize dokumentation 
    \begin{minted}
        [
            firstnumber=5,
            frame=lines,
            framesep=2mm,
            baselinestretch=1.2,
            linenos
            ]
    {java} 
    
    package com.tutego.insel.ds.observer;
    
    import java.util.*;
    
    public class Party
    {
      public static void main( String[] args )
      {
        Observer achim    = new JokeListener( "Achim" );
        Observer michael  = new JokeListener( "Michael" );
        JokeTeller chris  = new JokeTeller();
    
        chris.addObserver( achim );
    
        chris.tellJoke();
        chris.tellJoke();
    
        chris.addObserver( michael );
    
        chris.tellJoke();
    
        chris.deleteObserver( achim );
    
        chris.tellJoke();
      }
    }
    
    \end{minted}
        \label{lst:loesimpl}
    \end{listing}


    \subsubsection{Experimente Designen}

    Der Sinn hinter diesem Schritt ist es, eine Reihe von Experimenten zu designen, dessen 
    Resultate benutzt werden, um zu messen wie gut eine Aufgabe erfüllt wurde. Ein Experiment 
    erfasst Daten um den Erfolg einer konzipierten Lösung unter kontrollierbaren Bedingungen im Labor 
    zu testen. Vor dem Beginn der Experimente sollten alle aufgestellten und gesammelten Faktoren, 
    Regeln, Bedingungen und Beobachtungen auf Vollständigkeit und Korrektheit überprüft werden, da 
    jede dieser Komponenten einen fundamentalen Einfluss auf die Planung und Durchführung der 
    Experimente haben kann.  Wurde diese Liste noch ein letztes mal verifiziert,  kann nun mit 
    der Planungsphase für die Experimente fortgefahren werden. 
    Zuerst sollte sichergestellt werden, dass das Labor, in dem das Experiment stattfinden wird, 
    frei von nicht-essentiellen Objekten ist. Der negative Einfluss von emotionalen Reaktionen der 
    Testsubjekte sowie der Prüfer kann oft nicht vollkommen verhindert werden. Deshalb sollte man 
    darauf achten, dass jedes Element im Labor unabdingbar für das Experiment ist. Einmal richtig 
    angeordnet, sollte danach großer Wert darauf gelegt werden, dass niemand außer den am Experiment 
    beteiligten Personen das Labor betritt. Außenstehende können ohne ihr Wissen kleinste Veränderungen 
    am Arbeitsraum durchführen, z.B. ein Gerät um wenige Millimeter verschieben, und damit die 
    Ergebnisse des Experiments vollkommen aussagelos machen. Das schlimmste daran ist, dass ein 
    solcher Fehler, wenn überhaupt, nur sehr schwer zurückverfolgt werden kann, wodurch das ganze 
    Projekt in Gefährdung stehen könnte. Deshalb sollte das Labor in Abwesenheit des Personals 
    immer verriegelt und klar und deutlich für Außenstehende gekennzeichnet sein, um die Chance 
    auf derartige Fehler zu minimieren. Das eigentliche Experiment besteht aus mehreren Elementen. 
    Angefangen wird bei dem Personal, dass das Experiment durchführen wird. Diese haben einen 
    genauen Überblick über alle bisher gesammelten Daten sowie die nötige Expertise um die 
    Experimente auch durchzuführen. Lebendige Teilnehmer eines Experiments werden Kohorte genannt, 
    ihr Gegenpol sind die Stichproben. Die Durchführung eines Versuchs mit einer schrittweisen 
    Veränderung der Faktoren, bis alle möglichen Faktorkombinationen abgedeckt sind, heißt Block 
    Design. Um die vorgesehene Funktion aller Faktoren bestätigen zu können, werden sogen. Control 
    Trials durchgeführt. Dabei wird die Performance von einem Set von Faktoren in Abwesenheit eines 
    anderen Sets von Faktoren gemessen, um die isolierten Effekte des Sets festhalten zu können. 

    Die aus Experimenten erhaltenen Daten sollten auf mehr als einem Medium gespeichert sein, um 
    den Verlust im Falle eines Ausfalls eines der Medien zu verhindern. Datenschutz sollte eine der 
    obersten Prioritäten sein. Am besten sollten fertig bearbeitete Daten irgendwo im Internetz gespeichert
    werden, damit sie für jeden zugänglich sind. Dies könnte mit Hilfe einer Webseite, wie im Listing
    \ref{lst:htmlex} passieren. Es empfiehlt sich, die Daten zu verschlüsseln und diesen Schlüssel 
    nur an die Personen weiterzugeben, die unbedingt Zugriff auf die Daten benötigen. Werden über 
    die schon bestehenden Speicherplätze der Daten hinaus weitere Kopien angefertigt, so muss dies 
    unbedingt Dokumentiert werden, um unkontrollierte Verbreitung der Daten zu verhindern. Sind alle 
    Daten einmal gespeichert, sollten diese umgehend mit Hilfe des Copyright weiter geschützt werden. 
    In der Regel ist es sinnvoll, zu viele Daten über das Experiment gespeichert zu haben als zu wenig.   

    \begin{listing}[H]
        %caption wird benötigt, um das lisiting in das verzeichnis aufzunehmen, same mit abbildungen
        \caption[Anzeigemoeglichkeit Experimente via HTML]{Die Grundlage einer HTML Seite, zum Teilen der designten Experimente}
    % linenos -> numeriert zeilen, frame=lines -> grenzt code mit linien ab, fontsize -> schriftgröße, 
    % firstnumber -> zeile, in der snippet anfängt, muss al erster parameter stehen
    % unter den mathescape eingaben dann die gewählte sprache in {}
    % verfügbare sprachen via pygmentize dokumentation 
    \begin{minted}
        [
            firstnumber=5,
            frame=lines,
            framesep=2mm,
            baselinestretch=1.2,
            linenos
            ]
    {html} 
    
        <!DOCTYPE html>
        <html>
            <body>

                <h1>My First Heading</h1>
                <p>My first paragraph.</p>

            </body>
        </html>
    
    \end{minted}
        \label{lst:htmlex}
    \end{listing}

    \subsubsection{Experimente durchführen}

    Jetzt gilt es nur noch, die bereits fertig geplanten Experimente durchzuführen. Dies 
    sollte genauestens nach Plan passieren, damit Fehler oder unerwartete Ergebnisse in den 
    Plänen gefunden und behoben werden können. Misslungene Experimente sollten als Hilfen 
    gesehen werden, um die Pläne zu überarbeiten und dem Erfolg einen Schritt näher zu kommen. 
    Um eine fehlerfreie 


    \subsubsection{Ergebnisse extrahieren}

    Oft können die direkten Ergebnisse eines Experiments nicht genutzt werden, um 
    Schlussfolgerungen zu ziehen. Zuerst müssen die Resultate reduziert werden, also 
    umgewandelt und kombiniert, damit die entstehenden Werte in Form der gewählten 
    Performance Metrik dargestellt werden können. Wurden bis jetzt alle Richtlinien beachtet, 
    sollten die Daten über einen simplen SQL Befehl, gezeigt im Beispiel \ref{lst:sqlex} ohne 
    weitere Probleme extrahiert werden können. Im Grunde können nur zwei Arten von 
    Werten direkt gemessen werden: Distanzen und Zählungen. Weil die reduzierten Daten oftmals 
    modifiziert oder korrigiert werden müssen, sollten immer die rohen und die reduzierten 
    Ergebnisse des Experiments dokumentiert werden.


    \begin{listing}[H]
        %caption wird benötigt, um das lisiting in das verzeichnis aufzunehmen, same mit abbildungen
        \caption[SQL Befehl, zur Extraktion der Ergebnisse]{SQL Befehl, um Ergebnisse aus dem Experiment zu extrahieren}
    % linenos -> numeriert zeilen, frame=lines -> grenzt code mit linien ab, fontsize -> schriftgröße, 
    % firstnumber -> zeile, in der snippet anfängt, muss al erster parameter stehen
    % unter den mathescape eingaben dann die gewählte sprache in {}
    % verfügbare sprachen via pygmentize dokumentation 
    \begin{minted}
        [
            firstnumber=5,
            frame=lines,
            framesep=2mm,
            baselinestretch=1.2,
            linenos
            ]
    {sql} 
    
    USE AdventureWorks2012;
    GO
    SELECT *
    FROM Production.Product
    ORDER BY Name ASC;
    -- Alternate way.
    USE AdventureWorks2012;
    GO
    SELECT p.*
    FROM Production.Product AS p
    ORDER BY Name ASC;
    GO
    
    \end{minted}
        \label{lst:sqlex}
    \end{listing}
