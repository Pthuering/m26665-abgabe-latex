\section{Analyse}
(kursiv)
Die Aufgabe der Analysephase ist es, eingehendes Verständnis über alle Komponenten des Problemgebietes 
zu erlangen, sodass ein einzelnes, spezifisches und realistisches Ziel formuliert werden kann.

Das oberste Ziel der Analyse sollte immer sein, die vorerst viel umfassende und grobe Zielsetzung zu 
einer einzigen, spezifischen Aufgabe zu reduzieren. Die zuerst weite Problembeschreibung wird durch 
die Aufstellung leitender Thesen systematisch eingeschränkt, bis ein enges, aber hoch detailliertes 
Ziel im finalen Schritt der Analyse formuliert werden kann.

    \subsection{Beschreibung des Problems}

    Projekte können in zwei Untergruppen eingeteilt werden. Einerseits gibt es das Forschungsprojekt, 
    dessen Ziel die Gewinnung neuen Wissens ist. Daneben gibt es noch das Entwicklungsprojekt, bei dem 
    man bereits existierendes wissen nutzt, um ein neues Gerät zu entwickeln oder einen bestimmten 
    Effekt zu erzeugen. Häufig kann man ein Projekt allerdings nicht strikt zu einem der beiden 
    Gruppen zuordnen. Oftmals führt die Entwicklung eines neuen Geräts im Laufe seiner Entwicklungsphase 
    auch zu neuen Erkenntnissen. Manchmal ist es auch nötig ein neues Gerät zu entwickeln, damit neues 
    Wissen überhaupt zugänglich wird. Zweiteres trifft allerdings nur eher selten ein.

    Sollte es schwer fallen, das Problem mit einer kurzen, präzisen Frage oder Aussage zu umschreiben, 
    enthält das Projekt möglicherweise mehr als ein Problem. Dies ist ein häufig auftretender Grund für 
    Verwirrung im Projektplan, denn unterschiedliche Probleme benötigen fast immer auch unterschiedliche 
    Lösungen. Die dadurch verworrenen und sich teilweise widersprechenden Projektziele können zu großen 
    Komplikationen führen und verheerende Auswirkungen auf den Erfolg des Projekts haben, sollten sie 
    sich auch durch die Restlichen Projektphasen ziehen.
    Daher sollte am Anfang der Analyse eine möglichst detaillierte Zusammenfassung des Problems 
    angefertigt werden. Darin beinhaltet sind auch ausreichende Hintergrundinformationen und eine 
    Motivation zur Behebung des Problems, um das Projekt im richtigen Kontext einordnen zu können.

    \subsection{Performance Kriterien festlegen}

    Performance Kriterien sind Bedingungen, die jede vorgeschlagene Lösung zu einem Problem erfüllen muss. 
    Dabei sollte darauf geachtet werden, dass die Kriterien weder zu streng, noch zu offen sind. Das 
    Beispiel sei hier das Festlegen einer Deadline für einen Teil eines Projekts. Die vorgeschlagene 
    Lösung hier wäre drei Monate. Sind die Kriterien zu eng, würde das Projekt nach Ablauf der 
    festgelegten Zeit gestrichen werden, auch wenn die eigentliche Dauer gerade mal drei Monate und 
    ein Tag gewesen wären. Werden die Kriterien jedoch zu aussagelos gesetzt, könnte das Teilprojekt 
    potenziell nie ein Ende finden und das gesamte Projekt gerät in nicht aufholbaren Verzug. 
    Beide Ausgänge sind zu vermeiden.Die meisten Entwicklungsprojekte sind jedoch im Voraus schon von den 
    festen, strengen Bedingungen und Parametern der Anwendungsdomäne eingegrenzt.

    \subsection{Themenverwandte Arbeiten untersuchen}

    Ist das Problem dann formal und präzise beschrieben und alle grundsätzlichen Performance 
    Kriterien festgelegt, so kann das Projektteam nun damit anfangen, möglichst viele Informationen 
    über vorhergehende Projekte in einem ähnlichen oder im gleichen Rahmen zu sammeln. Der wohl 
    bedeutendste Grund dafür ist es zu verhindern, “Das Rad neu zu erfinden.” Sollten 
    schon Lösungen für Teile, oder sogar das Projekt im Ganzen bestehen, ist es fast immer 
    einfacher und billiger die Lösung oder das Gerät zu kaufen, als der Versuch es zu replizieren. 
    Es gibt verschiedene Arten, um nach relevanten Quellen zu suchen. Einige davon sind hier 
    zusammengefasst:

        (numerierte Liste)
        Professionelle Tagebücher
        Konferenzdokumentationen
        Bücher und Monographien
        Professionelle Studien 
        Datenbanken
        Zeitungsartikel
        Diskussionen mit Kollegen
        Reverse engineering

    Unabhängig von der Quelle muss die validität immer gründlich überprüft werden. 
    Zweitquellen, wie z.B. Zeitungsartikel, müssen immer auf ihre ursprüngliche Quelle 
    zurückführbar sein, um die Echtheit der Information versichern zu können. Die Quellen, die 
    als Vorwissen in ein Projekt gebracht werden, müssen klar und strukturiert zitiert werden.

    \subsection{Ziel formulieren}

    Dies ist der letzte Schritt in der Analysephase. Nachdem das Problem klar definiert, die 
    Performance Kriterien festgelegt und verwandtes Material gründlich untersucht wurde, ist die 
    Aufgabendomäne nun angemessen fokussiert und eingegrenzt. Jetzt kann ein spezifisches und 
    klares Ziel für das Projekt bestimmt werden. Das Ziel ist ein Ausdruck von dem, was das Projekt 
    im Bestfall erreichen soll. Bevor zur nächsten Phase übergegangen wird, müssen noch zwei 
    Faktoren beachtet werden. 
    Als erstes, das Ziel ist die Basis, an der Erfolg oder Versagen des Projekts gemessen 
    werden. Auch die Leistung des Projektteams wird daran gemessen, wie gut es das Ziel erreicht hat. 
    Als zweites, auch wenn andere Komponenten des Projekts relativ unkompliziert angepasst 
    werden können, benötigt es für eine Änderung des Ziels explizite Erlaubnis der Management 
    Ebene, was zur Verschwendung großer Zeit- und Geldressourcen führen kann und deshalb vermieden 
    werden sollte. Umso wichtiger ist es, schon von Anfang an ein allgemein anerkanntes Ziel festzulegen.

    
    

