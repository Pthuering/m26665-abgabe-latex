\section{Validation}

(kursiv) 
Das Ziel der Validation Phase ist es zu entscheiden ob eine optimale Lösung, in Abhängigkeit der 
Hypothese und Arbeitsmethodik,  erreicht wurde.

Die in der Hypothese festgelegten Metriken müssen flexibel sein. Da sich während der Problemlösung 
neue Informationen auftun können die mitbedacht werden müssen. Alle Ergebnisse die im Laufe 
verschiedener Test erarbeitet werden sind Teil eben dieser Informationen. Die Arbeit an einem 
Ingenieurs-Problem ist sich somit selbst ergänzend, da auf Basis dieser neuen Informationen 
Hypothesen erstellt, überarbeitet und verworfen werden. Die Validation ist somit sowohl fester 
Bestandteil als auch immer präsentes Mittel zur Ausarbeitung einer Lösung.

    \subsection{Wiederholtes Betrachten des Problems}

    Um eine Lösung als den Umständen entsprechend Optimal zu bezeichnen müssen wieder die 
    vier Schlagwörter aus dem Ersten Kapitel (Definition) zurückgreifen. Um zu bestimmen ob 
    die erarbeitete Lösung angemessen ist muss also geprüft werden ob:

        (Liste)
        Eine Veränderung stattfindet
        Diese Veränderung mit den gegeben Ressourcen möglich ist
        Diese Veränderung und die genutzten Ressourcen akzeptabel als Kompromiss sind
        Die unsicheren Informationen kein oder zumindest nur ein kleines Risiko beinhalten

    Die Prüfung dieser Aspekte mag trivial erscheinen, ist jedoch für die Entwicklung einer optimalen 
    Lösung essentiel. Das zu späte Bedenken oder gar Ignorieren dieser Schritte führt im besten Falle 
    zu Rückschritten im Projekt, könnte aber auch ein Scheitern eben dieses oder gar Schlimmeres mit 
    sich ziehen.

    Betrachten wir also die Konstruktion einer Brücke:

    Die Validierung der Veränderung ist denkbar einfach, nach dem Bau der Brücke ist es möglich 
    einen Fluss zu überqueren. Doch was ist wenn die erarbeitete Lösung nur eine Fußgängerbrücke 
    beinhaltet? In diesem Falle findet für alle Autofahrer keine Veränderung statt, wenn diese als 
    Teil der beteiligten Gesellschaft angesehen werden muss die vorgeschlagene Lösung folglich 
    abgelehnt werden.

    Auch die Validierung der Ressourcen scheint einfach. Sind alle nötigen Materialien, Fachpersonal 
    und genug Zeit vorhanden steht dem Bau der Brücke nichts mehr im Wege. Doch eben diese Ressourcen 
    können sich im Laufe der Planung ändern, es ist nicht mehr fest davon auszugehen, dass alle 
    geplanten Materialien zu dem geplanten Zeitpunkt verfügbar sind, dies muss als Teil der Lösung 
    geprüft werden. Ebenso können sich Personalanforderungen verändern oder Zeitfenster verschieben, 
    so ist eine Lösung vielleicht in der Theorie gut, doch in der Praxis zum Zeitpunkt der Umsetzung 
    gar nicht mehr möglich.

    Um zu Prüfen ob die vorgeschlagene Lösung die Beste ist wird ein ungemein großer Aufwand benötigt. 
    Dieser ergibt sich daraus, das im besten Falle alternative Lösungsansätze ausgearbeitet werden 
    deren Ergebnisse mit der ursprünglichen Lösung verglichen werden.Dies würde bedeuten, dass nicht 
    nur mehrere Brückenmodelle ausgearbeitet werden, sondern unter Umständen auch ein Tunnel als 
    Option aufgebracht wird. Da dies aber meisten nicht im Rahmen der Ressourcen ist werden 
    hilfsweise möglichst viele Faktoren betrachtet und ob deren Gewichtung innerhalb der Lösung 
    equivalent zu ihrer Gewichtung innerhalb der Problemstellung ist. Können also genug Fußgänger 
    und Autofahrer die Brücke überqueren? Wie sehr wird die Umwelt dadurch belastet? Könnte der 
    veränderte Verkehrsfluss Raststätten beeinflussen? Die Kriterien aller betroffenen Gruppen 
    werden hier nun erneut aufgewogen.

    Die Validierung der Ungewissheiten ergibt sich aus den vorangegangen Experimenten.  Es werden 
    bekannte Unsicherheiten minimiert und unbekannte Unsicherheiten entdeckt. 
    Um bekannte Unsicherheiten zu reduzieren werden in Experimenten eben diese Faktoren möglichst 
    fest bestimmt. Bei unserer Brücke ist zum Beispiel bekannt, dass der verbaute Stahl sich unter 
    Sonneneinstrahlung ausdehnt, doch müssen wir auch wissen wie weit er dies tut, wie schnell und 
    wie lange diese Verformung bestehen bleibt. Wenn diese Unsicherheiten durch Tests reduziert werden 
    können wir somit sicher sein, dass diese Problematik ressourceneffizient angegangen wurde. 

    Um unbekannte Faktoren aufzudecken werden Experimente und Tests unter realen Bedingungen 
    ausgeführt. Dadurch könnte zum Beispiel auffallen, dass das Wasser am Fuße der Brücke zu 
    Korrosion führt, ein Element welches vorher gar nicht bedacht wurde. Jetzt wo es bekannt 
    ist muss die dadurch entstandene Unsicherheit minimiert werden.  Für jedes Problem gibt es 
    unendlich viele Faktoren, Aufgabe der Tests ist es herauszufinden, welche ein realistisches 
    Risiko für den Lösungsansatz darstellen könnten.

    \subsubsection{Peer-Review}

    Nach all den Hypothesen und verworfenen Ideen, nach all den Experimenten und 
    gescheiterten Test steht nun endlich der Plan für eine Brücke. Alle bekannten Faktoren 
    sind bedacht, alle Ressourcen sind eingeplant, doch fehlt noch etwas. Dieses Etwas ist 
    es, was sowohl die Wissenschaftliche- als auch die Ingenieurmethodik revolutioniert hat. 
    Die Peer-Review.
    Wie schon in der Definition des Problems festgestellt ist die Gewichtung vieler Faktoren 
    trotz aller Bemühungen nie ganz Objektiv, daher ist es essentiell die Meinung und zustimmung 
    weiterer unbeteiligter Ingenieur einzuholen. Die Aufgabe dieser Ingenieure ist es die Gewichtung 
    der Faktoren zu Beurteilen, Verfahrensfehler zu finden und weitere Unsicherheiten und andere 
    Trugschlüsse auf zu decken. Auch unter der Annahme, dass nach bestem Wissen und Gewissen gehandelt 
    wurde sehen vier Augen immer mehr als zwei. Aus der Peer-Review können verschiedene Situationen 
    hervorgehen:
    
    Zum einen können sich Uneinigkeiten bei der Gewichtung der einzelnen Interessen ergeben. 
    Dies ist aufgrund mangelnder Objektivität meist schwer zu lösen, daher ist hier meist die 
    Meinung weiterer Unbeteiligter gefragt, woraus sich dann ein anwendbares Meinungsbild ergibt. 
    Außerdem können weitere Lösungsansätze hervorgehen, welche die gegebenen Ressourcen effizienter 
    nutzen, bessere Kompromisse von Interessen darstellen oder mehr Ungewissheiten ausschließen. 
    An dieser Stelle müssen nun die Ressourcen und Ungewissheiten betrachtet werden und und es 
    wird abgewogen, ob eine Änderung der Lösung und entsprechende Tests umsetzbar sind.
    Im besten Falle stimmen die zu Rate gezogenen Ingenieure natürlich der vorgelegten Lösung als 
    optimal zu, dann steht der Umsetzung nichts mehr im Wege.
