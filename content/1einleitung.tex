\section{Einleitung}

Der Begriff des Ingenieurs ist in vielen Berufsgruppen vertreten, begleitend von einer meist großen 
Menge sehr spezifischen Fachwissens. Darüber hinaus gibt es allerdings auch Methoden, die universell 
angewandt werden können. Im Grunde muss jeder Ingenieur dazu fähig sein, Situationen logisch zu 
analysieren, bestehende negative Wirkungen zu identifizieren, diese dann in strukturierte Probleme
 aufzuteilen und einen Lösungsweg für die Probleme zu erarbeiten. Auch die Innovation als großer 
 Bestandteil der Arbeit eines Ingenieurs, sei es das Erfinden neuer Geräte oder die Verbesserung 
 alter, lässt sich auf das Lösen einer langen Kette von Problemen herunterbrechen. Es gibt also 
 eine Vorgehensweise, die universell, ungeachtet der genauen Beschäftigung des Individuums, 
 angewandt werden kann. In den folgenden Abschnitten wird diese Methodik des Ingenieurs vorgestellt 
 und ihre einzelnen Schritte erklärt. 