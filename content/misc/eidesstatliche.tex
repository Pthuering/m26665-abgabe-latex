% vorlage für eine nicht unterschriebene eidesstattliche erklaerung
% kann bei bedarf einzeln exportiert, unterschrieben und als separate PDF an das hauptdokument angehangen werden

% * entfernt numerierung an titel, exkludiert es aber auch aus inhaltsverzeichnis
\section*{Eidesstattliche Erklärung}
% hinzufügen zu inhaltsverzeichnis, allerdings ohne numerierung
% statt section könnte auch subsection benutzt werden, für andere ordnung im verzeichnis 
\addcontentsline{toc}{section}{Eidesstaatliche Erklärung}
%abstand 
\vspace{10mm}

Ich erkläre hiermit an Eides statt, dass ich die vorliegende Arbeit selbständig verfasst und dabei 
keine anderen als die angegebenen Hilfsmittel benutzt habe. Sämtliche Stellen der Arbeit, die im 
Wortlaut oder dem Sinn nach Publikationen oder Vorträgen anderer Autoren entnommen sind, habe ich 
als solche kenntlich gemacht. Die Arbeit wurde bisher weder gesamt noch in Teilen einer anderen 
Prüfungsbehörde vorgelegt und auch noch nicht veröffentlicht.

\vspace{10mm}

Wernigerode, den 18.08.1999

% einfügen mathematischer formel mit dollar zeichen, workaround für unterschrift
% tilden für länge des striches

% orientierung der formel nach rechts
\begin{flushright}
    $ \overline{ ~~~~~\mbox{Philipp Thüring} ~~~~~} $ 
\end{flushright}
